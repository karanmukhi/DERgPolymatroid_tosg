\section{Introduction}

Power systems are increasingly integrating renewable energy resources to reduce environmental impact.
However, the intermittency and uncertainty of generation from wind and solar poses significant challenges. As the proportion of renewables rises, power system operators must adopt more flexible operational strategies to address these challenges. Meanwhile, with the electrification of heating and transportation, electricity demand is rising significantly \cite{Baruah2014EnergyElectrification}, and uncontrolled consumption from these new loads can strain the grid \cite{Lopes2011IntegrationSystem}. Many of these devices are inherently flexible, meaning there exist a set of consumption profiles that respect the operational constraints of the devices. For example, electric vehicles (EVs) are typically plugged in far longer than is necessary for a complete charge \cite{Lee2019ACN-Data:Dataset}, and thermostatically controlled loads (TCLs) can operate within a dead-band around their set-point temperature \cite{Callaway2009TappingEnergy}.
Distributed energy resources (DERs) encompass a diverse range of such small-scale loads and generators whose energy consumption can be actively managed.
Collectively, ensembles of these devices may offer substantial flexibility that can be used to mitigate the variability and uncertainty associated with renewable energy generation \cite{Almassalkhi2023IntelligentDecarbonization}.

Various control architectures have been proposed to make use of this flexibility, these can generally be categorized into centralized, decentralized and hierarchical schemes \cite{Callaway2011AchievingLoads}.
Fully centralized schemes, where devices report constraints to a system operator that solves a monolithic optimization and issues commands, are globally optimal but impractical and poorly scalable for large DER populations.
Decentralized approaches, such as those proposed in \cite{Tindemans2015DecentralizedResponse} and \cite{Gan2012OptimalCharging}, mitigate scalability challenges. However, they necessitate local computational capabilities in devices and may be unsuitable for real-time decision-making due to latency constraints. More seriously, the lack of a centralized decision-making entity makes it hard to assign accountability for maintaining strict grid reliability guarantees. Hierarchical control schemes can alleviate these issues, whereby an aggregator controls a population of DERs and presents its aggregate flexibility to the system operator \cite{Xu2016HierarchicalLoads}. In this scheme, the aggregator becomes the accountable party that can provide the reliability guarantees to system operators, whilst reducing the complexity for the system operator. An integral part of the function of an aggregator is to represent the aggregate flexibility in the population it controls.  

\subsection{Related Work}
Continuous-time exact representations of flexibility for energy storage systems (ESS) are proposed in \cite{Evans2020AResources, Zachary2021SchedulingStorage}. However,  these representations are ill-suited for the typical discrete-time operation of power systems. Indeed, most work considers the discrete-time variant of the problem, modeling consumption profiles as piecewise-constant functions. In this paradigm, the flexibility of a variety of DERs can be represented as a convex polytope that encode all operational constraints of the device \cite{Trangbaek2011ExactControl, Zhao2017ALoads, Hao2015AggregateLoads}. In some cases, the flexibility sets may be non-convex \cite{Taha2024WhenConvex}, however we will focus on consumption models that generate convex sets in this paper.
The aggregate flexibility of the population can then be characterized exactly as the Minkowski sum of these individual flexibility polytopes. In general, computing the Minkowski sum is difficult \cite{Tiwary2008OnPolytopes}, so most of the literature focuses on approximating the Minkowski sum. Efficient methods of computing outer approximations are proposed in \cite{Barot2017APolytopes, Wen2022AggregateModels}, however by definition these over-approximations include infeasible aggregate consumption profiles and therefore lack the reliability guarantees that are essential in power systems. Accordingly, most of the literature focuses on computing inner approximations of the sum:  many results focus on finding inner approximations of each flexibility set with a base polytope. These base polytopes are selected so that computing the Minkowski sum of a set of them is efficient, \cite{Kundu2018ApproximatingApproach, Muller2019AggregationResources,Alizadeh2014CapturingResponse, Taha2024AnPopulations}. In \cite{Muller2019AggregationResources} zonotopes are used,
whilst in \cite{Zhao2017ALoads} the authors describe a method of defining a base polytope finding inner approximations of the individual flexibility sets with homothets of this base. In \cite{Taha2024AnPopulations} this approach is generalized so that inner approximations may be affine transformations of the base set. Another scheme for deriving inner approximations is by viewing the Minkowski sum as a projection, and computing an inner approximation in the pre-projection space \cite{Zhao2016ExtractingApproximation}. \cite{Nazir2018InnerResources} forms an inner approximation as the union of a set of boxes: this can yield an arbitrarily good approximation for the polytope at the cost of increasing the computational burden. Inner approximations do not contain all feasible aggregate consumption profiles and so one cannot guarantee optimality when optimizing over them, indeed some of the approximations can be very conservative \cite{Ozturk2022AggregationAlgorithms}.
Clearly, only computing exact characterizations can guarantee optimality and feasibility, to this end \cite{Panda2024EfficientVehicles} and \cite{Mukhi2023AnVehicles} focus on a specific, but relevant, case of populations of EVs with charging only capability. These works characterize the individual flexibility sets as permutahedra and use properties of this class of polytopes to perform the Minkowski sum efficiently. The aggregate flexibility sets derived in \cite{Wen2022AggregateModels}, \cite{Mukhi2023AnVehicles}
 and \cite{Panda2024EfficientVehicles} are specific instantiations of the family of polytopes that we study in this paper, and the results in those works can be recovered from the more general theory presented here. 

 Finally, stochastic variants of the problem, where operational constraints of the individual devices are uncertain, have also been studied in \cite{Taheri2022Data-DrivenModels, Zhang2024AUncertainty}. Incorporating uncertainty is beyond the scope of this paper, however results from this paper can be applied to this extended setting \cite{Mukhi2025RobustFlexibility}. 
\subsection{Main Contributions}
 In this context, the contributions of this paper are summarized as follows:
\begin{itemize}
    \item We use generalized polymatroids (g-polymatroids) as a base polytope, and show that the flexibility set of a broad class of DERs can be represented exactly as a g-polymatroids. 
    \item Leveraging properties of g-polymatroids, we derive exact representations of the aggregate flexibility set for a population of DERs.
    \item For a population of  charging-only EVs, we show how these representations may be simplified further.
    \item Applying tools from combinatorial optimization we provide efficient methods of optimizing over these sets.
    \item Finally, we propose a vertex-based method to disaggregate an aggregate consumption profile among devices in the population.
\end{itemize}


The rest of this paper is structured as follows: in \cref{sec:prob_form} we introduce our DER consumption models and formalize the aggregation problem. \cref{sec:aggregation} begins with a brief overview of g-polymatroids and then proceeds to show how individual and aggregate flexibility sets may be represented as g-polymatroids. This is followed with a discussion on simplified representations for homogeneous populations.
In \cref{sec:optimization} we discuss how these sets may be efficiently optimized over, and \cref{sec:disaggregation} provides a vertex based disaggregation method. Numerical studies are presented in \cref{sec:numerical_results}, benchmarking the complexity and optimality of this work against competing methods. Finally, conclusions are drawn in \cref{sec:conc}. 


 \subsection*{Notation}\label{subsection:notation}
 \noindent
 For a vector $ u \in \mathbb{R}^{\mathcal{T}} $, where $ \mathcal{T} \subseteq \mathbb{N} $ is a finite index set and $ t \in \mathcal{T} $, we denote by $ u(t) $ the $ t^{\text{th}} $ component of $ u $.
 For any subset $ A \subseteq \mathcal{T} $,  we let $A' := \mathcal{T}\setminus A$.
 We also define $ u(A) := \sum_{t \in A} u(t) $.
 In particular, for $ t \in \mathcal{T} $, we write $ [t] := \{1, 2, \ldots, t\} $.
 Lastly, we use the notation $ \sum(\cdot) $ to denote both standard summation and Minkowski summation, with the specific meaning determined by context.

