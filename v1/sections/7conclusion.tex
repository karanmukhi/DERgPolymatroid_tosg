\section{Conclusions}\label{sec:conc}
This paper has proposed a novel method for characterizing the aggregate flexibility of distributed energy resource populations using g-polymatroids. 
We demonstrated that, under certain assumption, the flexibility sets of individual DERs can represented as g-polymatroids, and derived the corresponding super- and submodular functions. 
Using properties of this class of polytopes we demonstrated that their aggregate flexibility can be computed efficiently.
These representations are exact providing a significant advancement over existing methods, which primarily rely on inner or outer approximations.
Due to its exactness, the proposed approach ensures both optimality and feasibility in scheduling DER flexibility.
Furthermore, we showed how the super- and submodular functions representing the aggregate flexility of a population of EVs can be simplified thereby speeding up computations.
Finally, we developed efficient optimization methods over the aggregate flexibility sets and introduced a tractable disaggregation scheme.
Our computational results confirm that this approach is viable for large-scale DER aggregations.

Future research directions include extending this framework to incorporate stochastic elements, such as uncertainties in the operation constraints of the DERs.
Additionally, incorporating network constraints into the aggregation scheme could further refine its applicability. Integrating this approach with real-time control strategies may also enhance its practical implementation in power system operations. By leveraging the properties of g-polymatroids, this work provides a theoretical framework for more efficient DER coordination in power systems.